% ---------------------------------------------
%  describe document
% ---------------------------------------------

\documentclass[12pt
               % , final % turn on ``final'' to hide to-do bubbles
              ]{article}



% --- type, text, math, in/out encoding -----------------------

% This section is kinda non-essential
% These packages control typography, typeface, and input/output characters.
% The inputenc and fontenc pkgs may be important for non-English text,
%   but the rest are fluff.

\usepackage{microtype} % microtypographical things (not essential)

% fonts: 
\usepackage{mathptmx} % serif = times (with math fonts!)
\usepackage{helvet}   % sans-serif = helvetica clone
\usepackage{zi4}      % mono = inconsolata

% if you write non-English or need special accent characters, load these
\usepackage[utf8]{inputenc} % better interpretation of input characters
\usepackage[T1]{fontenc}    % better output glyphs/behaviors


% --- Margins and Spacing -----------------------

\usepackage[margin = 1.25in]{geometry} %margins % ipad geometry is 4X3
\usepackage{setspace} % line spacing
\usepackage{enumitem} % list settings
  \setlist{noitemsep} % (no separation between list items)


% --- math --------------------------------

\usepackage{amsmath} % math functions
\usepackage{amssymb} % math symbols


% --- tables and figures -----------------------

\usepackage{graphicx} % input graphics
\usepackage{float} % only good for H float option?
\usepackage{placeins} % for \FloatBarrier
\usepackage{booktabs} %? for toprule, midrule etc
\usepackage{dcolumn} % decimal-aligned columns



% --- document logic and utilities -----------------------

% ! not ``essential'' but handy enough to include

% hyperlinks and link colors
\usepackage{hyperref} 
\hypersetup{colorlinks = true, 
            citecolor = black, linkcolor = violet, urlcolor = teal}

% to-do notes
\usepackage[colorinlistoftodos, 
            prependcaption, 
            obeyFinal,
            textsize = footnotesize]{todonotes}
  \presetkeys{todonotes}{fancyline, color = violet!30}{}

% block comments
\usepackage{comment}





% --- References -----------------------

% you will need to change the path to the bib file
\usepackage[authordate, backend = biber]{biblatex-chicago}
\addbibresource{path/to/bib.bib}



% --- global title/section formatting -----------------------

% ! also non-essential
% These are aesthetic tweaks to the abstract and section title design.
% You may want to delete these if you don't like.

% abstract settings
\usepackage{abstract}
\renewcommand{\abstractname}{}    % abstractname replaced with NULL
\renewcommand{\absnamepos}{empty} % removes the block where abstract would
                                  %   have been placed, originally 'center'

% section title settings
\usepackage[rm, small, sc]{titlesec} % titles are non-bold, small, caps
\titleformat*{\subsection}{\itshape} % subsection titles are italic
\titleformat*{\paragraph}{\itshape}  % paragraph titles are italic










% ----------------------------------------------------
%   writing
% ----------------------------------------------------


\begin{document}


\title{Blank Template for Papers}
\author{Michael G.\ DeCrescenzo%
          \thanks{Ph.D.\ Candidate, University of Wisconsin--Madison. The author thanks the haters.}}
\date{Updated \today}
\maketitle


\begin{abstract}
  This document contains a preamble with my commonly used document settings. Many of the packages for math, spacing, and floats might be considered ``essential'' for social science papers. However, there are some stylistic changes to the type design (the type family, microtypography, section title designs) that are not essential. The source code highlights non-essential changes, so you should \emph{read the source code carefully before using this preamble} and modify it as you see fit. Lastly, some document parameters are for ``home use'' rather than article submission. The margin size (1.5 inch) and line spacing (1.5x) probably will need to be changed to conform to a journal's submission guidelines.
\end{abstract}



\onehalfspacing


\section{Introduction}

Lorem ipsum dolor sit amet, consectetur adipisicing elit, sed do eiusmod
tempor incididunt ut labore et dolore magna aliqua. Ut enim ad minim veniam,
quis nostrud exercitation ullamco laboris nisi ut aliquip ex ea commodo
consequat. Duis aute irure dolor in reprehenderit in voluptate velit esse
cillum dolore eu fugiat nulla pariatur. Excepteur sint occaecat cupidatat non
proident, sunt in culpa qui officia deserunt mollit anim id est laborum.


\subsection{Elaboration on a finer point}

Let's say we wanted to write some pretty math. Let's say we really thought item response theory (IRT) models were beautiful, and we were to write one flavor of IRT model:

\begin{align}
  \mathrm{Pr}\left( y_{ij} = 1 \right)
  &=
  \Phi \left[
    \beta_{j} \left( \theta_{i} - \alpha_{j} \right)
  \right] \\
  \theta_{i} &\sim \mathrm{Normal}\left(\bar{\theta}_{g[i]}, \sigma_{g[i]} \right)
\end{align}


\newpage
\printbibliography


\end{document}
