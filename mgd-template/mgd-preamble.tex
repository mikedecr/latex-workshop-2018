% ---------------------------------------------
%  describe document
% ---------------------------------------------

\documentclass[12pt
               % , final
              ]{article}



% --- type, text, math, in/out encoding -----------------------
\usepackage{microtype}

% minion pro loads textcomp, MnSymbol, amsmath
% if you want to pass options, load them beforehand
\usepackage[lf, mathtabular, minionint]{MinionPro} % serif familiy
\usepackage{MyriadPro} % sans-serif family
% \usepackage{zi4}

% if minion and/or myriad fail, load these
% \usepackage{mathptmx} % serif = times (with math)
% \usepackage{helvet} % sens-serif = helvetica clone

\usepackage[utf8]{inputenc} % better interpretation of input characters
\usepackage[T1]{fontenc}    % better output glyphs/behaviors


% --- Margins and Spacing -----------------------

\usepackage[margin = 1.25in]{geometry} %margins % ipad geometry is 4X3
\usepackage{setspace}
\usepackage{enumitem} % allows nosep option for compact lists


% --- tables and figures -----------------------

\usepackage{graphicx} % input graphics
\usepackage{float} % only good for H float option?
\usepackage{placeins} % for \FloatBarrier
\usepackage{booktabs} %? for toprule, midrule etc
\usepackage{dcolumn} % decimal-aligned columns



% --- document logic and utilities -----------------------

% hyperlink options
\usepackage{hyperref} 
\hypersetup{colorlinks = true, 
            citecolor = black, linkcolor = violet, urlcolor = teal}

% to-do notes
\usepackage[colorinlistoftodos, 
            prependcaption, 
            obeyFinal,
            textsize = footnotesize]{todonotes}
  \presetkeys{todonotes}{fancyline, color = violet!30}{}

\usepackage{comment} % block comments





% --- References -----------------------

\usepackage[authordate, backend = biber]{biblatex-chicago}
\addbibresource{/Users/michaeldecrescenzo/Dropbox/bib.bib}



% --- global title/section formatting -----------------------

% \usepackage{titling}
% we used to have title customization stuff but it's noisy

\usepackage{abstract}
\renewcommand{\abstractname}{}    % clear the title
\renewcommand{\absnamepos}{empty} % originally center

\usepackage[rm, small, sc]{titlesec}
\titleformat*{\subsection}{\itshape}
\titleformat*{\paragraph}{\itshape}



% --- user commands -----------------------

% notes for figures
\newcommand{\notes}[1]{\hfill \\
% \raggedright 
\small
\emph{Notes:} #1}

% input with unskip as one command
\newcommand{\pull}[1]{\input{#1}\unskip}








% ----------------------------------------------------
%   writing
% ----------------------------------------------------


\begin{document}


\title{Blank Template for Papers}
\author{Michael G.\ DeCrescenzo%
          \thanks{Ph.D.\ Candidate, University of Wisconsin--Madison. The author thanks the haters.}}
\date{Updated \today}
\maketitle


\begin{abstract}
  Abstract for a paper about some stuff. We took away the abstract title because it isn't necessary. For the text block, we use microtype, wider margins than one-inch, 1.5x line spacing. Type styles use the Minion and Myriad families (which will probably fail on most computers, sike!), and a font identity that uses no bolding---the only text decoration is small capitals and italics. We don't need huge section titles either, so we shrink 'em.
\end{abstract}

\vspace{12pt}


\onehalfspacing


\section*{Introduction}

Lorem ipsum dolor sit amet, consectetur adipisicing elit, sed do eiusmod
tempor incididunt ut labore et dolore magna aliqua. Ut enim ad minim veniam,
quis nostrud exercitation ullamco laboris nisi ut aliquip ex ea commodo
consequat. Duis aute irure dolor in reprehenderit in voluptate velit esse
cillum dolore eu fugiat nulla pariatur. Excepteur sint occaecat cupidatat non
proident, sunt in culpa qui officia deserunt mollit anim id est laborum.


\subsection*{Elaboration on a finer point}

Let's say we wanted to write some pretty math. Let's say we really thought item response theory (IRT) models were beautiful, and we were to write one flavor of IRT model:

\begin{align}
  \mathrm{Pr}\left( y_{ij} = 1 \right)
  &=
  \Phi \left[
    \beta_{j} \left( \theta_{i} - \alpha_{j} \right)
  \right] \\
  \theta_{i} &\sim \mathrm{Normal}\left(\bar{\theta}_{g[i]}, \sigma_{g[i]} \right)
\end{align}



\newpage
\printbibliography


\end{document}
